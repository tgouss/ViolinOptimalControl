\part{Introduction}
    \chapter{Stage international}

Mon cursus à l'\'Ecole des Mines de Saint-Étienne me permet au deuxième semestre de deuxième année de réaliser un \emph{projet industriel} sur le campus ou un \emph{stage international}.
J'ai souhaité partir à l'étranger, l'expérience associée me semblant davantage complète et engageante. La dimension professionnelle du stage, mettant en jeu des interactions sociales différentes de celles rencontrées au sein d'associations et de groupes de projets étudiants en est la première raison. Mon choix s'est porté sur le Canada et la province du Québec. Principalement motivé par le laboratoire de simulation et modélisation du mouvement : \emph{s2m}, la culture québecoise et les splendides paysages ne m'ont que confortés dans ce choix.

Mme Camille \textsc{Wilhelm}, étudiante de la promotion précédente ayant réalisé un stage international au laboratoire s2m m'en a fait une présentation globale : activités de recherches, état d'esprit, encadrement et méthodes de travail. J'ai ensuite échanger avec M. Benjamin \textsc{Michaud}, ancien tuteur de Mme  Camille \textsc{Wilhelm}. La discussion a porté sur ses activités de recherches, ses besoins et mes compétences.
Cette série d'échange, entre autre, m'a permis de prendre conscience de l'intérêt de s'essayer à un travail de recherche et non industriel, dans l'optique d'appréhender fidèlement ces deux mondes.


    \chapter[Laboratoire s2m]{Laboratoire simulation et modélisation du mouvement}
        \section{Histoire}
        
En 2008, M. Mickaël \textsc{Begon} fonde le laboratoire de Simulation et Modélisation du Mouvement dont l’objectif est la recherche en biomécanique et la formation de personnel hautement qualifié en kinésiologie. D'abord installé dans les locaux du Centre de réadaptation Marie-Enfant à Montréal, le laboratoire migre à Laval en 2011, lors de la fondation du
nouveau campus de l’université de Montréal.


        \section{Domaine d'expertise}
        
Le laboratoire s2m est porté vers le développement de nouvelles connaissances sur la motricité humaine à
partir de mesures et de modèles de simulation pour des applications en réadaptation, prévention des
blessures et amélioration de la performance sportive et artistique. On peut citer des projets phares comme l'optimisation dynamique d'acrobaties, l'optimisation du geste violonistique et pianistique, ou encore la conception d’orthèses plantaires personnalisées. Les projets se rapportent tous à une thématique musicale, sportive ou ergonomique.
Le laboratoire S2M fonctionne majoritairement grâce aux subventions des organismes du
Québec et du Canada comme la fondation Canadienne pour l’innovation ou l’Institut de recherche en
Santé et Sécurité au travail. Il est aussi en relation avec des partenaires industriels. % toujours vrai ?


Le laboratoire bénéficie d’équipement de pointe pour les mesures biomécaniques : 
\begin{itemize}
\setlength\itemsep{-0.5em}
\item Un ergomètre isocinétique.
\item Un système optoélectronique de 18 caméras.
\item Un système EMG : Électromyographie (intramusculaire et de surface).
\item Des plateformes de force sur une piste de marche.
\item Un tapis instrumenté pour la marche.
\item Un piano acoustique.  
\end{itemize}
\newpage

    
        \section{Effectif}      
Supervisé par M. Mickaël \textsc{Begon} et M. Fabien \textsc{Del Maso}, le laboratoire de recherche universitaire s2mlab peut s'apparenter à un outil pédagogique de formation étudiante. L'effectif de recherche regroupe en moyenne 25 chercheurs. Il se renouvelle très régulièrement, allant du stagiaire de quelques mois à la thèse ou au doctorat de plusieurs années, à titre d'exemple, on peut dénombrer 6 arrivants lors de mon stage. L'objectif de ce renouvellement est de proposer à un maximum d'étudiants de vivre une expérience scientifique afin d’insuffler un goût pour la recherche, amenant certains à faire suite à leur stage avec une thèse ou un doctorat.
\begin{figure}[h]
\begin{center}
\begin{tikzpicture}[
    every node/.style = {shape=rectangle, rounded corners, draw, align=center, top color=white, bottom},
    level 1/.style={sibling distance=60mm},
    level 2/.style={sibling distance=30mm},
    level 3/.style={sibling distance=25mm}]]
    \node {S2mlab\\\footnotesize{(25 pers)}}
        child { node {Permanents}
            child { node {Assistants de\\recherche\\\footnotesize{(5)}} }
            child { node {Coordinateurs\\\footnotesize{(5)}} } }
        child { node {Temporaires}
            child { node {\'Etudiants}
                child { node {Postdoctoral\\\footnotesize{(5)}} }
                child { node {Thèse\\\footnotesize{(5)}} }
                child { node {Maitrise\\\footnotesize{(5)}} } } };
\end{tikzpicture}
\caption{Organigramme du laboratoire}
\end{center}
\end{figure}
\label{equipe_control_optimal}
Le laboratoire est découpé en plusieurs thématiques, en tant que stagiaire de Benjamin \textsc{Michaud}, j'ai rejoint l'équipe de \emph{optimal control}, \emph{commande optimale} en français. Composé d'une dizaine de chercheurs, l'équipe se réunit tous les mardi matin lors d'un tour de table où chacun expose ses avancées et ses problèmes. L'aspiration de cette réunion est double, débloquer rapidement les chercheurs qui sont en difficultés et s'assurer d'une collaboration efficace : jouir de l'expérience des autres chercheurs pour ne pas essayer de réinventer l'eau chaude. M. Mickaël \textsc{Begon}, professeur agrégé, participe à cette rencontre et apporte, entre autre, une vision biomécanique et un sens de l'efficacité.



        \section{Recherche scientifique}
        
Il est important de différencier la recherche scientifique de la recherche industrielle. Si la première se veut collaborative et nécessite le concours de plusieurs laboratoire, la deuxième est par nature concurrentielle afin de commercialiser en premier des technologies et de déposer des brevets.

%revenir sur la collaboration avec l'idée de se comparer articles BoirbdOptim (MOCO)

Un chercheur scientifique se doit donc de publier des articles afin de partager ses recherches à ses homologues. A travers ses articles, et les citations de ses articles, le chercheur peut appuyer ses demandes de financement. La recherche, n'engendrant pas de profit direct, nécessite de faire appel à des acteurs extérieurs. La collectivité assume en grande partie ce rôle en finançant les universités, qui finance ensuite le laboratoire s2mlab. %à vérifier !

%open source ??


    \chapter{Contexte}
        \section{Covid-19}
        
        
Lors de mon arrivé le 15/03/20, l'université de Montréal a fermé ses portes à la totalité de ses étudiants, professeurs et chercheurs. Les locaux du laboratoire s2m ont ainsi été fermés, privant les chercheurs de matériel expérimental et de certaines ressources informatiques.

        \section{Télétravail}

L'obligation de télétravail pour l’ensemble des chercheurs du laboratoire a nécessité la mise en place de plusieurs outils informatiques :

\begin{description}
\setlength\itemsep{-0.5em}
\item[Messagerie :] \emph{\gls{slack}} puis \emph{\gls{teams}}.
\item[Visioconférence :] \emph{\gls{zoom}}, comprenant une fonctionnalité de \emph{remote control} : prise de contrôle à distance de l'ordinateur.
\end{description}

En complément de ceux existants :

\begin{description}
\setlength\itemsep{-0.5em}
\item[Gestion de développement de logiciels :] \emph{\gls{github}} couplé avec \emph{\gls{gitkraken}}.
\item[Répartition de taches :] \emph{trello}.
\item[Partage de fichiers :] \emph{serveurs du laboratoire}.
\end{description}

En écartant \emph{trello} dont l'utilisation est rendue caduc par le système d'\emph{issue} proposé par \emph{github}, on peut considérer que 3 logiciels de communication permettent de répondre aux besoins émanants de collaboration.

\begin{table}[h]
\begin{center}
\begin{tabular}{|c|c|c|c|c|}
\hline
Mode & Temps & Type & \'Echéance & Logiciel\\
\hline
oral & long ($\simeq$ heure) & réflexions stratégiques/développement & long terme & \emph{zoom} \\
\hline
écrit & court ($\simeq$ min) & aide rapide, mémorisation par écrit  & court terme & \emph{microsoft teams} \\
\hline
écrit  &  instantané & fichiers de programmation (\textbf{code}) & court terme & \emph{github}, \emph{gitkraken} \\
\hline
écrit  &  court ($\simeq$ min) & liste d'idées (\textbf{issue}) & long terme & \emph{github}, \emph{gitkraken} \\
\hline
\end{tabular}
\end{center}
\end{table}

S'il est pertinent de remarquer qu'un échange oral permet un débat profond et efficace autour d'une stratégie de développement, on peut regretter l'éphémérité de l'information. Il est possible d'y palier en inscrivant l'information essentiel sur teams, sur les issues de github ou au sein des lignes de codes.

La répartition des échanges est simple, néanmoins, elle implique, à chaque instant, de réfléchir au moyen de communication le plus adapté.

Avec du recul, il est aisé de détecter un choix de communication inadapté, par exemple une discussion sur teams avec un nombre de messages important, laisse présager qu'une vidéoconférence aurait été davantage efficace. En revanche, une vidéoconférence sans prise de note, c'est à dire n'aboutissant pas à une issue, un message ou un commentaire dans le code, laisse présager une perte d'information.

Enfin, il faut garder à l'esprit qu'un outil de collaboration est là pour économiser du temps et préserver l'information utile, ce qui revient à économiser du temps. L'objectif étant de se concentrer davantage sur la recherche. Il faut ainsi toujours veiller à rester vigilant et lucide face à l'utilisation faite des outils de communications.


        \section{Construction d'équipe}
        
Le confinement étant de vigueur, les régulières activités de groupe du laboratoire ont été suspendues. Malgré cela, certaines activités ont pu se réinventer via logiciel de vidéoconférence :
\begin{itemize}
\setlength\itemsep{-0.5em}
\item Séance de sport.
\item Séance de yoga.
\item Escape-game en ligne.
\end{itemize}

Au fur et à mesure de l'assouplissement du confinement, des activités extérieures de groupe ont pu voir le jour, dans le respect des distances et des gestes barrières :
\begin{itemize}
\setlength\itemsep{-0.5em}
\item Séances de travail collectives en appliquant la technique \emph{\gls{pomodoro}} dans des parcs.
\item Randonnées et séances sportives.
\end{itemize}

La finalité de ces temps partagés avec des collègues chercheurs, est de casser la monotonie de télétravail depuis la maison. En télétravail, il est particulièrement complexe de dissocier les temps de travail des temps personnels. D'autant plus dans un cadre de recherche où l'avancement, et ainsi l'entrain varient. Il en résulte que mon temps de travail hebdomadaire est en moyenne supérieur à 35 heures par semaine, avec un temps de travail quotidien oscillant entre 6h et 9h.

Il s'est avéré que le simple fait de travail en visioconférence avec ses collègues, sans forcement discuter, permet au travers du bruit de fond et de leur présence de reconstituer un contexte de travail.
