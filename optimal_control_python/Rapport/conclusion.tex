\part{Conclusion}
    \chapter{Conclusion}
Ces quatre mois et demi de télétravail m'ont permis de co-développer un logiciel générique de contrôle optimal que j'ai parallèlement utilisé pour modéliser et optimiser la gestuelle du violoniste.

    \subsection*{BiorbdOptim}
    
Le développement de \emph{BiorbdOptim} m'a offert une expérience de travail collaboratif à distance. Progressivement, j'ai intégré le rôle et l'utilisation des divers outils de communication. J'ai compris, par la pratique et l'échec, qu'il est primordial de toujours savoir situer son travail par rapport au groupe et au projet. Sans cela, il devient inévitable de programmer des fonctionnalités déjà existantes ou incompatibles avec le reste du projet. Dans ce sens, la présence d'un manager assurant la cohérence des développements demeure irréfragable. Enfin, l'entraide étant permanente et réciproque au sein de mon groupe de travail, je ressors de ce stage convaincu par l'adage \emph{``seul, on va plus vite, ensemble on va plus loin''}. J'en tire la conclusion qu'il est essentiel de réfléchir avec lucidité face à chaque obstacle~: faut-il chercher seul, demander de l'aide et à qui, ou retarder la confrontation ? Tout cela nécessite de l'expérience, ce stage a contribué, à sa hauteur, à m'en pourvoir.


Le développement de \emph{BiorbdOptim} m'a également offert une expérience de création de logiciel avec le langage Python, en y intégrant un logiciel écrit avec le langage C++. Outre les compétences indéniablement acquises en programmation, j'ai utilisé et joué avec un logiciel de gestion de projet informatique : \emph{github} renforcé par \emph{gitkraken}. Le terme ``joué'' me semble important, car c'est en expérimentant, que l'on acquière une maîtrise complète d'un logiciel.
De même, l'utilisation quotidienne du système d'exploitation \emph{Linux} et de \emph{miniconda} m'a permis de manipuler ces outils basiques avec aisance. Enfin, si je ne devais retenir qu'une morale, alors je choisirais sans hésiter l'importance d'écrire de tests, dès le commencement, qui vérifient l’exactitude de toutes les lignes de code et de leurs évolutions.

\emph{BiorbdOptim} s'étoffe progressivement, en intégrant continuellement plus de fonctionnalités. \'A ce jour,  il égal son principal concurrent : \emph{MOCO}. Un article est, à ce jour, en écriture afin de présenter \emph{BiorbdOptim} à la communauté scientifique de contrôle optimal appliqué à la biomécanique. Le propos de cet article est de comparer ses performances et fonctions à ceux d'une référence : \emph{MoCo}.
Je suis fier d'avoir écrit les premières lignes de ce logiciel, et d'avoir contribué, approximativement, à l'écriture d'un quart du logiciel.


    \subsection*{Optimisation de la gestuelle du violoniste}
    
La gestuelle de violoniste, peu étudiée par les biomécaniciens,...

modélisation, importante, 74 états 18 muscles 10 dof
xia
multiphase tant d'aller retour
intéret de BiorbdOptim rapide à coder

démarche scientifique

difficulté de dissocier la démarche scientifique et le développement de BiorbdOptim (qu'on outil)

J'aurais souhaité...

Il reste encore à...
%perspectives résultantes du travail du stagiaire
%les enseignements tirés du stage
%les réussites et les échecs du stage/projet

    
    \subsection*{Expérience générale}
    
L'expérience apportée par ces 4 mois et demi passés virtuellement au laboratoire s2m correspond à mon attente : découvrir le métier de chercheur. Je peux succinctement, citer divers apprentissages~: s'il est évident de mener ses propres recherches, il demeure crucial de chercher les conclusions de ses homologues. J'ajouterais la valeur de considérer avec recul les situations de stagnation, où il parait insurmontable de parvenir à son objectif. C'est là tout l'enjeu de la recherche : expérimenter ce que personne n'a déjà tenté.
Subséquemment à la frustration de la stagnation, la joie enivrante lors d'un succès alimente la volonté de perpétuer et perfectionner ses travaux. Bien que n'ayant qu'à traverser mon appartement pour me rendre sur mon lieu de travail, je l'ai toujours fait avec alacrité et excitation.

Finalement, l'expérience apportée par ces 4 mois et demi passés effectivement à Montréal me remplie de satisfaction. En dépit du contexte pandémique, j'ai pu profiter de l'accueil enthousiaste des québecois à commencer par le tutoiement qui l'illustre parfaitement. Je retiendrais des belles rencontres avec de longs échanges sur l'histoire et la double colonisation du Québec, dont la devise est : \emph{Je me souviens, que né sous le lys, je croîs sous la rose}. Au terme de mon stage, j'ai pu jouir de la superbe nature québecoise, de ses lacs et forêts, peu apprivoisées par l'homme.

% comprendre qu'on est tous différent au sein du labo et france/quebec
% complément école/asso
