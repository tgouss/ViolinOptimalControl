\newpage
\addcontentsline{toc}{part}{Résumé}
\textbf{\huge Résumé}
\vspace{0.5cm}\\\tab
Le présent rapport s'intègre au sein d'une thèse portant sur l’optimisation du mouvement du violoniste afin de minimiser la fatigue musculaire. Ma contribution à ce projet fut parallèlement partagée entre le développement d'un outil générique de contrôle optimal : \emph{BiorbdOptim}, et l'optimisation du mouvement du violoniste grâce à cet outil.\emph{BiorbdOptim} est une interface entre un utilisateur et un solveur de problèmes de contrôle optimal. Cet outil est développé par et dans une première finalité, pour les chercheurs du laboratoire \emph{s2mlab}. L'utilisation de \emph{BiorbdOptim} et d'un modèle, issue de la librairie \emph{BIORBD}, des membres supérieurs humains portant un violon et son archet, m'a permis d'écrire un problème de contrôle optimal. Ce problème s'est progressivement raffiné, en intégrant successivement : le mouvement de tiré-poussé via des couples articulaires, le maintient du contact entre l'archet et la corde, le respect du parallélisme entre l'archet et le chevalet, le maintient du même angle pour l'archet, une force de contact entre l'archet et le violon, les muscles aux alentours du bras, une modélisation de fatigue musculaire, plusieurs mouvement successifs de tiré-poussé. %todo à compléter
La finalité de cette modélisation est de parvenir à déterminer le plus fidèlement possible, et par du contrôle optimal le mouvement minimisant la fatigue musculaire dans le but d’aider à l’enseignement du violon.

\vspace{1.5cm}
\textbf{\huge Abstract}
\vspace{0.5cm}\\\tab
This report is part of a thesis on the optimization of the violinist's movement in order to minimize muscular fatigue. My contribution was shared between the development of a generic optimal control tool: \emph{BiorbdOptim}, and the optimization of the violinist's movement thanks to this tool. \emph{BiorbdOptim} is an interface between a user and a solver of optimal control problems. Further, it is developed by and also for the researchers of the s2mlab laboratory. The use of BiorbdOptim and a model of human upper limbs carrying a violin and its bow, from the \emph{BIORBD} library, allowed me to write an optimal control problem. This problem has been progressively enhanced, successively integrating these concepts : the back and forth movement via articular couples, maintaining contact between the bow and the string, respecting parallelism between the bow and the bridge, maintaining the same angle for the bow, a contact force between the bow and the violin, the muscles around the arm, a model of muscular fatigue, several successive pull-push movements. %todo to be completed
The purpose of this modeling is to determine as accurately as possible, and through optimal control, the movement that minimizes muscle fatigue in order to assist in the teaching of the violin.





